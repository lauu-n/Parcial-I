# Implementación del promedio en cálculo lambda

El promedio de una lista de `n` números se define como:

\[
\text{promedio}(xs) = \frac{\text{suma}(xs)}{\text{longitud}(xs)}
\]

---

## Definición en cálculo lambda

En notación de cálculo lambda pura podemos escribirlo como:

\[
\text{Promedio} = \lambda xs.\; \frac{\Sigma(xs)}{|xs|}
\]

donde:

- \(\Sigma(xs)\) es una función que calcula la suma de los elementos de la lista.
- \(|xs|\) es una función que calcula la longitud de la lista.

---

## Ejemplo

Si \(xs = [2,4,6]\):

\[
\text{Promedio}([2,4,6]) 
= \frac{\Sigma([2,4,6])}{|[2,4,6]|} 
= \frac{2+4+6}{3} 
= 4
\]

---

## Interpretación

- La función `Promedio` es una **función anónima** (`λ xs`) que recibe una lista.
- Se aplica la función `Σ` para obtener la suma de los elementos.
- Se aplica la función `|xs|` para obtener la longitud.
- Finalmente, se divide la suma entre la longitud para obtener el promedio.
